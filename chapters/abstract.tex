\begin{abstract}

本项目将人工智能技术与传统物理实验结合,设计实现了一套基于YOLO目标检测算法的单摆实验智能测量系统。该系统通过AI计算机视觉技术实现对单摆运动的精确追踪,自动采集数据并进行分析处理,有效解决了传统人工测量中精度低、操作繁琐等问题,并且与利用传统计算机视觉技术的Tracker软件相比,本系统在复杂背景下实现了更高精度的测量。系统由高精度单摆实验装置和自主开发的AI辅助软件组成,软件集成了三种互补的周期测量方法(峰值检测法、FFT频谱分析法和曲线拟合法),实现了重力加速度和阻尼系数的自动测量。实验结果表明,系统测得的重力加速度为$(9.785 \pm 0.319)$ m/s$^2$ (k=2),与当地标准值相比平均相对误差仅为0.19\%;在本项目的实验环境下,单摆系统对空气阻尼的阻尼系数为$(7.5 \pm 0.17) \times 10^{-5}$ N$\cdot$s/m (k=2),均显著优于传统测量方法。系统通过AI智能化辅助实现了低成本、易操作和直观可视化等优势,可广泛应用于物理实验教学中,为学生提供更精确、高效的实验体验,实现了AI技术与物理教育的深度融合。

\keywords{YOLO目标检测;单摆实验;重力加速度;阻尼系数;计算机视觉;物理实验教学}
    
\end{abstract}

