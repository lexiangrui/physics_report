\section{创新点}
\begin{MainBox}[方法创新——AI智能识别与多模式数据处理相结合]
\quad\quad 本项目首次将YOLO目标检测算法应用于单摆实验中,开发出一套能够实时、精准识别摆球位置的AI辅助系统。该方法突破了传统计算机视觉方法的局限,能够适应不同光照条件和背景干扰,提高小球定位精度,实现了计算机视觉技术与物理实验的深度融合。
并创新性地设计了三种互补的周期测量方法(FFT频谱分析法、峰值检测法与曲线拟合法),综合运用了频域和时域分析技术,极大地提升了周期测量的精确性和稳定性。
\end{MainBox}


\begin{MainBox}[功能创新——多参数同步测量技术助力物理实验教学]
\quad\quad 传统的单摆实验通常只测量一个物理量,而本系统实现了单次实验同时获取周期、阻尼系数和振幅变化等全部动力学参数,大幅扩展了单摆实验的测量能力。通过多种可视化图表(轨迹图、位置-时间图、频谱分析图、对数衰减图等)直观展示物理过程,可在实验教学中帮助学生形成对周期运动和阻尼振动概念的深入理解。
\end{MainBox}


