\section{结论与展望}
\subsection{结论}

本项目基于YOLO深度学习目标检测算法,成功设计并实现了一套用于物理实验的智能自动化测量系统,实现了人工智能技术与传统物理实验的深度融合。通过系统实验与分析,本团队得出如下结论。

研究证明,\textbf{YOLO目标检测算法可以成功应用于物理实验中的动态目标追踪,实现对实验对象的精确定位与参数测量}。系统采用的三种互补周期测量方法(峰值检测法、FFT频谱分析法与曲线拟合法)结果高度一致,验证了系统的稳定性与可靠性。与传统人工测量方法相比,AI辅助测量显著提高了数据采集精度;与Tracker软件相比,本系统无需人工标定,并能在复杂背景下实现更高精度的测量。在重力加速度测量实验中,五组不同摆长的综合分析显示,最终测得的重力加速度平均值为$g = (9.785 \pm 0.319)$ m/s$^2$ (k=2),与武汉地区标准值$9.7936$ m/s$^2$相比,最佳测量结果的相对误差仅为0.02\%,\textbf{平均相对误差为0.19\%}。在阻尼系数测量方面,系统获得的本实验的阻尼系数为$(7.5 \pm 0.17) \times 10^{-5}$ N$\cdot$s/m (k=2),根据参考文献\textsuperscript{\cite{WUJS202211018}}理论分析,本实验测量结果与该单摆系统的实际阻尼系数相近。

不确定度分析表明,系统测量精度受到多种因素影响,主要包括:相机帧率限制导致的周期测量不确定度(贡献了重力加速度总不确定度的约99.1\%)以及YOLO目标检测的位置精度。重力加速度测量的合成相对不确定度为1.63\%,阻尼系数测量的合成相对不确定度为1.16\%,均显著优于传统手动测量方法。理论分析和实验结果均表明,当振幅比接近3.03时,阻尼系数的测量不确定度最小,这一发现为提高阻尼特性测量精度提供了重要指导。

系统实现了从视频数据采集到结果输出的全流程自动化处理,消除了人为操作误差,简化了实验步骤,显著提高了实验效率。系统基于普通单摆装置和常规摄像设备,无需额外专业仪器,大幅降低了实验成本,提高了技术推广的可行性。此外,图形化界面与数据可视化功能使物理过程直观呈现,有助于学生对周期运动和阻尼振动概念的理解,完成了AI技术与物理实验的有机结合。


\subsection{展望}

虽然本项目已取得初步成果,但在技术完善和应用拓展方面仍有广阔的发展空间。未来研究可从技术优化、应用拓展等方向展开深入探索。

在技术优化方面,本团队计划进一步扩充训练数据集规模,增强数据的多样性,引入更多光照条件、背景复杂度和摆球特征的变化,提升YOLO模型在复杂环境下的识别稳定性,同时探索模型轻量化方案,减少计算资源需求。本团队还将优化算法实现,提高视频处理和分析的实时性能,实现实验过程中的即时数据反馈,使系统不仅能用于事后分析,也能用于实时监测与教学演示。特别地,通过提高相机帧率至120 FPS,理论上可将重力加速度测量的总相对不确定度降低至约0.88\%,提升测量精度约46\%,这将是未来系统优化的重要方向。

在应用拓展方面,本系统可以适配到更多基础物理实验中,如自由落体、弹簧振子、碰撞实验、波动现象等,构建一个涵盖力学、电磁学、光学等领域的综合AI辅助实验平台,进一步探索AI实验平台与课程教学的融合。此外,基于本项目建立的不确定度评估体系,可以进一步完善物理实验中的测量精度分析方法,为学生提供更加全面的实验数据处理与误差分析训练。