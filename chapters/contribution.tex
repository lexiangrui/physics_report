\section{学生个人贡献}
本实验项目由来自数学与统计学院、物理与力学学院及信息学院的五位本科生组成的跨学科团队共同完成,在指导教师的悉心指导下,历时六个月完成了从系统构想到实验验证的完整研究流程。团队成员在人工智能、数据处理与物理实验等方面各有所长,形成了较为合理的知识结构与分工协作机制。

项目准备阶段(12 月至次年 1 月),团队对大学生物理实验竞赛的内容与评审标准进行了系统调研,明确了以“基于人工智能辅助的单摆周期测量系统”为研究主题。通过查阅相关文献与技术资料,确定采用基于 YOLO 模型的目标检测算法结合图像分析手段进行周期自动提取的可行技术路线。

项目初期(2 月至 3 月),团队完成了初代实验平台的搭建,实现了高质量视频采集、YOLO 模型的部署与测试,并构建了像素级运动轨迹提取模块,初步验证了图像识别在摆运动测量中的有效性。

项目中期(3 月至 4 月),着重开发了数据处理与周期分析模块,涵盖信号平滑、峰值检测、周期计算与拟合分析等关键算法,实现了系统对实验数据的自动分析处理。期间团队同步进行多组物理场景实验拍摄与数据测试,反复调试以提升系统鲁棒性与分析精度。

项目末期(4 月至 5 月),团队在前期工作的基础上对各模块功能进行了集成与优化,确保系统运行稳定、输出准确。同时,完成了实验报告的撰写与 LaTeX 文档排版,制作了完整的系统演示视频,最终形成了具备展示性、实用性与创新性的竞赛项目成果。

\begin{table}[H]
\centering
\caption{学生个人贡献表}
\begin{tabular}{@{}c p{12.5cm}@{}}
\toprule
\textbf{队员编号} & \multicolumn{1}{c}{\textbf{个人贡献}} \\
\midrule
1 & 项目负责人,负责整体统筹与任务分配,设计AI辅助实验系统方案。主要负责YOLO模型的训练与数据预处理,包括图像标注、模型调参与结果验证,并撰写关键技术文档与实验报告。 \\
2 & 主要负责论文撰写与排版,整理技术内容并规范表达,协调各模块内容整合为完整文本,同时美化数据图表,确保论文质量与表达规范。 \\
3 & 协助完成YOLO模型训练、部署与输出数据处理,参与图像标注与周期提取精度分析,后期协助视频拍摄与剪辑,提升项目展示效果。 \\
4 & 负责实验平台搭建与调试,包括摆锤结构设计、视频采集设置及相机校准,保障数据采集稳定性,同时参与PPT制作与展示材料整理。 \\
5 & 负责数据处理模块开发,实现周期提取相关算法,并与YOLO输出有效对接,提升系统自动化水平,协助论文代码内容的逻辑校对与验证。 \\
\bottomrule
\end{tabular}
\label{tab:contribution}
\end{table}